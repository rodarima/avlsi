\documentclass[a4paper]{article}
%\usepackage{amsfonts}
\usepackage{amsmath}
%\usepackage{amsthm}
\usepackage[utf8]{inputenc}
%\usepackage{hyperref}
%\usepackage{booktabs}
\usepackage{graphicx}
\usepackage{subfig}

\title{Exercises on physical design}
\author{Rodrigo Arias Mallo}
\date{\today}

\newcommand*\mat[1]{ \begin{pmatrix} #1 \end{pmatrix}}
\newcommand*\arr[1]{ \begin{bmatrix} #1 \end{bmatrix}}
\newcommand*\V[1]{ \boldsymbol{#1}}

\begin{document}
\maketitle

\section{Quadratic placement}

% A = [[ 3 -1 -1  0]
%  [-1  3  0 -1]
%  [-1  0  2  0]
%  [ 0 -1  0  2]]
% bx = [ 2.  0.  4.  4.]
% by = [ 0.  4.  3.  3.]
% X = [ 2.28571429  1.71428571  3.14285714  2.85714286]
% Y = [ 1.76190476  2.9047619   2.38095238  2.95238095]

By computing the Laplacian matrix of the graph, and removing the rows and 
columns of the fixed cells, we get the matrix $A$ used in the linear equation 
system.
%
$$ A = \mat{
	 3 & -1 & -1 &  0 \\
	-1 &  3 &  0 & -1 \\
	-1 &  0 &  2 &  0 \\
	 0 & -1 &  0 &  2}$$
%
The right hand side for $x$ is $b^x = \mat{2 & 0 & 4 & 4}^T$ and for $y$ is $b^y 
= \mat{0 & 4 & 3 & 3}^T$, so we can solve now the two systems: $AX = b^x$ and 
$AY = b^y$ and obtain the coordinates of the non-fixed cells.
%
$$ X = \mat{2.29 & 1.71 & 3.14 & 2.86}^T,\quad
Y = \mat{1.76 & 2.90 & 2.38 & 2.95}^T $$
%
The graph can be plotted now, where the black nodes are the fixed ones 
$\{u,v,w\}$ while the gray nodes are the non-fixed cells $\{a,b,c,d\}$. For 
details see the python script \texttt{placement.py}.
%
\begin{center}
\includegraphics[width=.7\textwidth]{graph.pdf}
\end{center}
%
\section{Channel routing}

\end{document}
